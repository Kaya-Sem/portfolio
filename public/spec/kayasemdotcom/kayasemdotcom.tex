\documentclass[10pt,a4paper]{article}
\usepackage[utf8]{inputenc}
\usepackage[margin=1in]{geometry}
\usepackage{xcolor}
\usepackage{graphicx}
\usepackage{titlesec}
\usepackage{enumitem}
\usepackage{hyperref}
\usepackage{amsmath}
\usepackage{listings}
\usepackage{tikz}
\usetikzlibrary{shapes,arrows,positioning}

% Color scheme
\definecolor{primary}{RGB}{41,128,185}
\definecolor{secondary}{RGB}{52,152,219}
\definecolor{accent}{RGB}{231,76,60}
\definecolor{lightgray}{RGB}{245,245,245}
\definecolor{darkgray}{RGB}{51,51,51}

% Custom section formatting
\titleformat{\section}
  {\normalfont\Large\bfseries\color{primary}}
  {\thesection}{1em}{}

\titleformat{\subsection}
  {\normalfont\large\bfseries\color{secondary}}
  {\thesubsection}{1em}{}

% List styling
\setlist[itemize]{leftmargin=*,itemsep=0.1cm,topsep=0.2cm}
\setlist[enumerate]{leftmargin=*,itemsep=0.1cm,topsep=0.2cm}

% Code listing style
\lstset{
    basicstyle=\ttfamily\small,
    breaklines=true,
    frame=single,
    backgroundcolor=\color{lightgray},
    rulecolor=\color{darkgray},
    numbers=left,
    numberstyle=\tiny\color{darkgray},
    keywordstyle=\color{primary},
    commentstyle=\color{gray},
    stringstyle=\color{accent},
}

% Header and footer
\usepackage{fancyhdr}
\pagestyle{fancy}
\fancyhf{}
\rhead{\color{darkgray}\thepage}
\lhead{\color{darkgray}\leftmark}
\renewcommand{\headrulewidth}{0pt}

% Project title command
\newcommand{\projecttitle}[1]{
    \begin{center}
        {\Huge\bfseries\color{primary} #1}\\
        \vspace{0.5cm}
        \textcolor{darkgray}{\large Project Documentation}
    \end{center}
}

\begin{document}

% Title Page
\begin{titlepage}
	\centering
	\vspace*{2cm}

	\projecttitle{kaya-sem.com}

	\vspace{1cm}
	\begin{center}
		\href{https://github.com/Kaya-Sem/portfolio}{\includegraphics[width=0.1\textwidth]{../images/github-mark.png}}
	\end{center}

	\vspace{1cm}

	\textcolor{darkgray}{\large Kaya-Sem Van Cauwenberghe}\\
	\textcolor{darkgray}{\today}

	\vfill

	\begin{abstract}
		\noindent
		\textcolor{darkgray}{An interactive portfolio website featuring a force-directed graph visualization that showcases my projects and their technological interconnections. Each project node links to comprehensive LaTeX-generated documentation, allowing visitors to explore the technical details and design decisions behind each project.}
	\end{abstract}

	\vspace{1cm}

	\small
	\textcolor{darkgray}{Document version: 1.0}
\end{titlepage}

\newpage

\section{Project Overview}

\subsection{Key Features}
\begin{itemize}
	\item \textbf{Force-Directed Graph}: Projects and technologies as interactive nodes in a physics simulation
	\item \textbf{Project Documentation}: LaTeX-generated PDF specifications for each project
\end{itemize}

\section{Technology Choices}
\subsection{The Stack}
\begin{itemize}
	\item \textbf{Svelte}: Open-source and simpler compared to other frameworks. While I would have preferred to avoid JavaScript entirely, D3 integration necessitated its use.
	\item \textbf{D3.js}: The core visualization library, doing most of the heavy lifting
	\item \textbf{PDF.js}: For seamless documentation viewing
\end{itemize}

\section{Reflections}

The website is overkill for my usecase, but I am madly in love with graphs and semantic systems. For larger datasets, this may be more interesting. I imagine myself mapping the connections between GitHub projects, or creating an interactive visualization of the fashion world - linking designers, fashion houses, and runway shows.

\end{document}
